\documentclass{article}

\usepackage[T2A]{fontenc}
\usepackage[utf8]{inputenc}
\usepackage[russian]{babel}
\usepackage[a4paper, total={7in, 10in}]{geometry}
\usepackage{hyperref}
\usepackage{enumitem}
\newcommand{\nt}[1]{{\color{red} [#1]}}
\newcommand{\ntt}[1]{{\color{green} [#1]}}
\hypersetup{
    colorlinks=true,
    linkcolor=blue,
    filecolor=magenta,      
    urlcolor=blue
}

\title{Papers RL}

\begin{document}

\maketitle

\section{Models from papers}

\begin{itemize}

    \item \href{https://papers.nips.cc/paper/714-feudal-reinforcement-learning}{"Feudal Reinforcement Learning", Dayan, Hinton, NIPS-1992, 320 citations}. В стате рассматривается такая задача \ntt{заполню}. В данной задаче сравниваются 2 модели: классический Q-learning и авторский Feudal. 
\end{itemize}

\section{Datasets}

\begin{itemize}

    \item  Classic atari games. \ntt{Надо написать про то какая sota и хотя бы примерно как она берется}
    \item Minecraft. \ntt{сейчас конкурс на этот "датасет"\ на нипс. Кажется хорошо подходит}
\end{itemize}

\section{Unsorted}

\begin{itemize}
    \item \href{https://papers.nips.cc/paper/6233-hierarchical-deep-reinforcement-learning-integrating-temporal-abstraction-and-intrinsic-motivation.pdf}{Hierarchical Deep Reinforcement Learning}
    \item \href{}{} 
    \item \href{}{} 
    \item \href{}{}
\end{itemize}

\end{document}
